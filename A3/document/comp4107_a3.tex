\documentclass[11pt]{article}
\usepackage{fullpage,amsmath,mathtools, algorithm2e, forest}
\usepackage[mathletters]{ucs}
\usepackage{hyperref}
\usepackage[utf8x]{inputenc}
\usepackage{graphicx}
\usepackage{listings}
\usepackage{courier}

\lstset{basicstyle=\footnotesize\ttfamily,breaklines=true}
\lstset{frame=single}

\graphicspath{ {./images/} }
\title{COMP4107 - Assignment 3}
\author{Student Name: Yunkai Wang\\
\text{Student Number: 100968473}\\\\
Student Name: Jules Kuehn\\
\text{Student Number: 100661464}}
\date{Fall 2018}
\begin{document}
\maketitle
\begin{enumerate}

\item
a) See q1.py for code.\newline
\newline
As shown in the diagram and table below, the fewer training images were used, the better the Hopfield network performed in general. This suggests that a Hopfield network is a very poor choice for classifying even two classes.\newline
A major factor in the accuracy was the selection of training images. Since fewer training images performs better, with a random selection the performance of the network is basically a matter of chance. To address this, we selected the training images with k-means, taking the images closest to the cluster centers, and as many cluster centers as training images. This would (in general) make for training images that are more different from each other, which would limit the degeneration.
\begin{figure}[h!]
    \centering
    \begin{minipage}{0.45\textwidth}
        \centering
        %\includegraphics[width=0.9\textwidth]{q1a-2} % first figure itself
        \caption{Function Contours}
    \end{minipage}\hfill
    \begin{minipage}{0.45\textwidth}
        \centering
        %\includegraphics[width=0.9\textwidth]{q1a-2-costs} % second figure itself
        \caption{Cost (MSE) at epoch}
    \end{minipage}
\end{figure}\newline
Using 8 neurons for the hidden layer is appropriate for this function. The MSE for the test data with 8 neurons is better that is required, and the training converges after fewer epochs than the network using 50 neurons for the hidden layer.\newline
The table below corresponds to the experiment shown in the graph above.\newline
%\lstinputlisting{python_output/q1a-results.txt}


b) By setting "storkey = True" in q1.py, you can inspect our results for the bonus.\newline
\newline
As shown in the table below, the Storkey algorithm does not substantially improve the results for the classification.

\begin{figure}[h!]
    \centering
    \begin{minipage}{0.45\textwidth}
        \centering
        %\includegraphics[width=0.9\textwidth]{q1b-2b} % first figure itself
        \caption{Convergence vs optimizer}
    \end{minipage}\hfill
    \begin{minipage}{0.45\textwidth}
        \centering
        %\includegraphics[width=0.9\textwidth]{q1b-3_cpu-time} % second figure itself
        \caption{CPU time vs optimizer}
    \end{minipage}
\end{figure}

The reduced number of required training epochs for RMSPropOptimizer (seen in the table below) more than compensates for the increased CPU time for each epoch, seen in Figure 4.\newline
%\lstinputlisting{python_output/q1b-results.txt}


c) We can further show that 8 neurons is a good choice by noting that early stopping has no effect. Our experiments have not been able to demonstrate an overtraining phenomenon with even 50 hidden neurons, but a premature early stop can be seen in Figure 6, and the table below.\newline
%\lstinputlisting{python_output/q1c-results.txt}

\begin{figure}[h!]
    \centering
    \begin{minipage}{0.45\textwidth}
        \centering
        %\includegraphics[width=0.9\textwidth]{q1c-1-50_neurons_slow_convergence2} % first figure itself
        \caption{Slow convergence}
    \end{minipage}\hfill
    \begin{minipage}{0.45\textwidth}
        \centering
        %\includegraphics[width=0.9\textwidth]{q1c-2-earlyStopping} % second figure itself
        \caption{Premature early stop}
    \end{minipage}
\end{figure}

The network with 50 hidden neurons also is very slow to converge to the target MSE - even using the RMSPropOptimizer - compared to when using 8 hidden neurons (Figure 5).\newline

\begin{figure}[h!]
    \centering
    \begin{minipage}{0.45\textwidth}
        \centering
        %\includegraphics[width=0.9\textwidth]{q1c-3} % first figure itself
        \caption{Reaching training goal}
    \end{minipage}\hfill
    \begin{minipage}{0.45\textwidth}
        \centering
        %\includegraphics[width=0.9\textwidth]{q1c-32} % second figure itself
        \caption{Early stopping not triggered}
    \end{minipage}
\end{figure}

Figure 7 shows the MSE at each epoch near where the training goal is reached. This captures the expected trend - that the training error will decrease consistently, while the test error (9x9 grid) has more variation. The validation data is most different from the training data (as it is randomly sampled from the function), so it is expected that its error would also have the highest variance.\newline
\newline
Figure 8 shows that having an early stopping mechanism for this function makes no difference, as the early stopping mechanism is never triggered (8 hidden neurons, 4000 epochs, 0.02 LR). Thus, the two contours are virtually identical, differing only by the stochastic elements of the experiments (initial weights, order of training data). 

\pagebreak
\item Instructions on how to run q2.py:\newline
\newline
Running q2.py without changing anything will run all three parts. This may take a while as it runs 10 experiments for each number of hidden neurons, and after one is finished, a new neural net is created to run the second one, etc. Therefore, it will take about 1 min for it to finish and plot the graphs. To run a single part only, comment out the other lines in the experiment() function.\\
We started by using the default learning rate for AdamOptimizer (0.001). This took 400-700 epochs to converge to the result. However, if we use 0.01 as the learning rate, it converges faster -- within 100 epochs every time. Therefore, we chose 0.01 as the learning rate for this question.\\

% \lstinputlisting{python_output/q2.out}
% \includegraphics[width=0.7\columnwidth]{q7_result3}
\begin{enumerate}
\item We took the averages of 10 experiments for each of 5, 10, 15, 20, and 25 hidden neurons. The results are shown below.\newline

\begin{figure}[h!]
    \centering
     %\includegraphics[width=0.5\textwidth]{q2-a-1}
        \caption{Percentage of Recognition Errors}
\end{figure}

The table below corresponds to the experiment shown in the graph above.\newline
%\lstinputlisting{python_output/q2a-results.txt}

\item 15 is chosen as the number of hidden neurons, since 15 is more accurate than 10, but using more than 15 neurons is not much better (if at all). When we used 15 neurons in the hidden layer, the number of epochs to reach zero errors is much less than is shown in the assignment question, which we think is fine. Perhaps this is due to a different learning rate.\\
\begin{figure}[h!]
    \centering
    \begin{minipage}{0.45\textwidth}
    	\centering
        %\includegraphics[width=0.9\textwidth]{q2-b-1} % first figure itself
        \caption{Training on perfect data}
    \end{minipage}\hfill
    \begin{minipage}{0.45\textwidth}
    	\centering
         %\includegraphics[width=0.9\textwidth]{q2-b-2} % first figure itself
         \caption{Re-training after noisy data}
    \end{minipage}\hfill
\end{figure}

\item The network trained with both perfect and noisy data performed just as well on perfect data, and better on noisy data. The chart below is for the network with 15 neurons.\\
\begin{figure}[h!]
    \centering
     %\includegraphics[width=0.5\textwidth]{q2-c}
        \caption{Percentage of Recognition Errors}
\end{figure}

\end{enumerate}

\pagebreak
\item Question 3

\begin{enumerate}
\item For part a, here are the parameters and dimension of the SOM computed. We used only 1024 samples to train the SOM, the reason for doing this is just to shorted the running time. Otherwise the SOM takes too long to converge. Since we are using 1024 samples, so we used a SOM of size $12 \times 12$. The $\sigma$ value is $1$, and the learning rate is $0.1$. Here is the graphs generated by using these parameters:\\
\begin{figure}[h!]
    \centering
    \begin{minipage}{0.45\textwidth}
    	\centering
        \includegraphics[width=0.9\textwidth]{Before_training}
        \caption{Training on perfect data}
    \end{minipage}\hfill
    \begin{minipage}{0.45\textwidth}
    	\centering
         \includegraphics[width=0.9\textwidth]{After_training_for_250_epochs}
         \caption{Re-training after noisy data}
    \end{minipage}\hfill
    \begin{minipage}{0.45\textwidth}
    	\centering
         \includegraphics[width=0.9\textwidth]{After_training_for_500_epochs}
         \caption{Re-training after noisy data}
    \end{minipage}\hfill
\end{figure}
It's clear that as time proceed, the SOM starts to group all the 1's together, and all the 5's together, where initially the SOM was initialized randomly.

\item Here are the plots for the K-means solutions with different number of clusters.
\begin{figure}[h!]
    \centering
    \begin{minipage}{0.45\textwidth}
    	\centering
        \includegraphics[width=0.9\textwidth]{Without_kmeans}
        \caption{Training on perfect data}
    \end{minipage}\hfill
    \begin{minipage}{0.45\textwidth}
    	\centering
         \includegraphics[width=0.9\textwidth]{With_2_clusters}
         \caption{Re-training after noisy data}
    \end{minipage}\hfill
    \begin{minipage}{0.45\textwidth}
    	\centering
         \includegraphics[width=0.9\textwidth]{With_3_clusters}
         \caption{Re-training after noisy data}
    \end{minipage}\hfill
    \begin{minipage}{0.45\textwidth}
    	\centering
         \includegraphics[width=0.9\textwidth]{With_4_clusters}
         \caption{Re-training after noisy data}
    \end{minipage}\hfill
    \begin{minipage}{0.45\textwidth}
    	\centering
         \includegraphics[width=0.9\textwidth]{With_4_clusters}
         \caption{Re-training after noisy data}
    \end{minipage}\hfill
\end{figure}

\end{enumerate}

\newpage
\item Question 4
\begin{enumerate}
\item We tried this question with two hidden layers, and we tried a different number of hidden layers for both layers. But the accuracy doesn't change for the different number of layers that we tried. The accuracy is always $9\%$ for all values that we tried. There may be some problem with the implementation we had.
\begin{figure}[h!]
    	\centering
        \includegraphics[width=0.9\textwidth]{q4}
        \caption{Training on perfect data}
\end{figure}


\end{enumerate}

\end{enumerate}
\end{document}